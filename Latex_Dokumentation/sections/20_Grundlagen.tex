\chapter{Grundlagen}

\section{JDroid Library}
\sectionauthor{\philipp}

Die wichtigste Komponente der Kartenspiele ist die JDroid library. Die von
\href{http://www.aplu.ch/home/apluhomex.jsp?site=99}{Äegidius Plüss} entwickelte
Java Bibliothek, implementiert ebenfalls die JCardGame Bibliothek für Android,
welche grundlegend für unsere Kartenspiele ist. Sie verfügt über eine
ausführliche
\href{http://www.java-online.ch/gamegrid/index.php?inhalt_links=navigation.inc.php&inhalt_mitte=iframedoc1.html}{Dokumentation}.
Im folgenden erkläre ich die wichtigsten von uns genutzten Funktionen der
Bibliothek.

\subsection{Rank und Suit enum}

Zu Anfang muss man die Farben und Wertigkeiten der Karten in Form eines Enums
festlegen. Karten werden durch sprites dargestellt, die man in drawables ablegt. Durch die
Namensgebung der Sprites, ordnet die Bibliothek die Sprites der richtigen Karte
zu. 
Angenommen die Reihenfolge der Ranks lautet:

Ass, Ober, unter, Zehn, koenig, usw.

und die der Suits sei:

Eichel, Gruen, Herz, Schellen

,wäre der korrekte Name für den Sprite für die Eichel Ass: eichel0.gif, für den Eichel Ober: eichel1.gif, und so weiter.
\subsection{Locations}

Um Hände und Kartenstapel anzeigen zu können, muss man zunächst im Konstruktor
ein Board erstellen, und Ausrichtung, Farbe, sowie windowZoom(int) angeben. Der
windowZoom unterteilt den Bildschirm in Zellen relativ zur Bildschirmgröße, um
darauf sogenannte Locations anzulegen, auf welchen man Karten ablegen kann oder
Textfelder.

Am beispiel Schafkopf haben wir 3 Location Arrays genutzt:

\begin{itemize}
\item HandLocations (Für alle 2er Kartenstapel)
\item StackLocations (Ablage Stapel für gestochene Karten)
\item BidLocations (Stapel um einen Stich zu berechnen)
\end{itemize}

\subsection{Einstiegspunkt und initPlayers}

Anders als normal ist der Einstiegspunkt nicht in \code{OnCreate()}, sondern wurde
durch die Bibliothek überschrieben und startet in \code{main()}. Dort werden das Deck
auf Basis der enums, sämtliche Locations, die Hände und die Spieler
initialisiert . In \code{initPlayers()} läuft das Spielgeschehen ab, es wird dem
Spieler der soeben an der Reihe ist der CardListener hinzugefügt und
\code{longPressed(Card card)} wartet darauf, dass eine Karte gedrückt gehalten wird.
Wurde eine Karte ausgewählt wird sie auf den jeweiligen bid transferiert und
\code{atTarget()} wertet den Stich dann aus und/oder gibt einen marker für den aktiven
Spieler mittels der Methode \code{setPlayerMove(int playerWon)} weiter, welche jeweils
\code{.setTouchEnable(}) der einzelnen Kartenstapel auf true oder false setzt.

\subsection{Deck und Hands}

Das Deck wird initialisiert aus allen Suits und Ranks, die man Anfangs innerhalb
der enums festlegt. Zudem legt man ebenfalls fest wie die Rückseite der Karten
aussehen soll. Anschließend wird mit einer vorgegebenen Methode gemischt und
ausgeteilt.

\code{dealingOut(int AnzahlHände, int AnzahlKarten, Boolean Mischen)}

Eine Hand beinhaltet Kartenobjekte und verwaltet diese. Sprich Stack, bids und
die Spieler Hände bestehen alle aus Hand Objekten, die auf bestimmten Locations
angezeigt werden. Mittels \code{stacks[i].setView(board, new StackLayout(stackLocations[i])} legt man diese Locations für jede Hand
einzeln fest und zeigt sie mit \code{stacks[i].draw()} an.

\subsection{Textactor}

Textactor sind Textfelder welche man auf bestimmten Locations anzeigen lassen
kann. Da sie sowohl im Schafkopf als anzeige für die verbleibenden Karten, als
auch als Punkteanzeige verwendet werden, ist es wichtig sie an dieser Stelle zu
erläutern. Zuerst muss ein Textactor mit einem String initialisiert werden, zum Beispiel mit der Anzahl der Karten des ersten 2er Stapels. Desweiteren
braucht jeder Actor eine Location. Mit \code{addActor(TextActor, Location)} zeigt man
den Actor auf dem Board an und mit \code{removeActor(TextActor)} wird er wieder
entfernt.

