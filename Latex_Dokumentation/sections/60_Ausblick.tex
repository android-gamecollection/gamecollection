\chapter{Ausblick}
\chapterauthor{\frank}

Trotz des sehr großen Erfolges unseres Programmierprojektes, gibt es immer noch
Sachen, die man vielleicht doch noch zu unserer Applikation hinzufügen möchte.
Oder vielleicht gibt es ja Sachen, die es aufgrund der relativ kurzen Zeit nicht
hinein geschafft werden. Ein paar dieser Dinge und Ideen werden im Folgenden
aufgelistet.

\section{Module}

Mit unserem Projekt lassen sich ziemlich einfach neue Spiele hinzufügen. Alles,
was man dafür tun muss ist, das Spiel in \code{games.xml} einzutragen und die
benötigten Klassen und Resourcendateien hinzuzufügen. Auch die Vererbung von der
Klasse \code{GameActivity} erleichtert die ganze Sache ungemein.\\ Doch ein
großer Nachteil bleibt: Um ein neues Spiel hinzufügen zu können braucht man
immer noch Einsicht und Zugriff auf den Quellcode der Applikation. Wie wäre es,
wenn man Spiele einfach von einem Repository herunterladen könnte, um sie der
App hinzuzufügen? Das Format hierfür würde einer API entsprechen, die ein Spiel
ansprechen kann. Man kann es sich ähnlich wie eine Spielemodifikation für
Minecraft oder Factorio -- wer es kennt -- vorstellen. Das Spiel wird als
ZIP-Datei in einen bestimmten Ordner eingefügt und kann dann beim Starten der
Applikation automatisch aus diesen Dateien geladen und dem Spielmenü hinzugefügt
werden. Die Struktur dieser Datei selbst würde aus einer Meta-Beschreibung des
Spiels und dem eigentlichen Spielinhalt bestehen. Diese Idee kann mit einem
zentralen Online-Repo und InApp-Downloadmöglichkeit verknüpft werden.

\section{Onlinespiele}

Aktuell ist man limitiert auf das Spielen an einem Gerät. Ob gegen den Computer
oder gegen den besten Freund gegenüber, beides ist möglich. Doch was ist, wenn
mal keinen Kumpel in der Nähe hat und die KI einem irgendwann langweilig wird?
Für den, der lieber im Keller sozial ist, wäre die Lösung: ein asynchroner
Onlinemodus. Hierfür werden die Spielstände zusammen mit Meta-Informationen, wie
etwa den beteiligten Spielern auf einem Server gespeichert. Anschließend kann
der gegnerische Spieler die entsprechenden Daten des Servers bei sich
herunterladen und den nächsten Zug ausführen. Das wäre ein ziemlich großes
Unterfangen, denn man braucht nicht nur die ganze Logik, die dahinter steckt,
sondern auch die Infrastruktur mit Datenbank und Wartungsarbeiten -- wie etwa
veraltete Spielstände zu löschen. Der Umfang hier wäre wieder in etwa so groß
wie die Dauer eines gesamten Programmierprojektes.