\chapter{Einleitung}

\section{Spiele auf dem Smartphone}
\sectionauthor{\leonard}

... sind heutzutage gang und gäbe. Ob kleine Mini-Spiele oder umfangreichere mit
Tiefe, sie alle sind begehrt. Auch Sammlungen kleinerer Spiele erfreuen sich
großer Beliebtheit. Doch muss man für diese meist schon viel Arbeit in die
Menüs und Navigation stecken. Das kann sehr viel Zeit in Anspruch nehmen; Zeit,
die eigentlich in die Implementierung der Spiele selbst investiert werden
sollte.

\section{Ein Gerüst für Spielesammlungen}
\sectionauthor{\leonard}

Deshalb haben wir ein Gerüst geschaffen, welches das Hinzufügen von Spielen
erleichtert. Zudem sind in der App auch exemplarisch einige Karten- und
Brettspiele implementiert, um die Handhabung zu veranschaulichen.

\section{Das Schachspiel}
\sectionauthor{\frank}

Gleichzeitig haben wir uns zum Ziel gesetzt, eine voll funktionsfähige
Implementierung eines Schachspieles zu erschaffen. Man soll sowohl gegen eine
KI, als auch gegen einen weiteren menschlichen Spieler lokal antreten können.
Die Schwierigkeit sollte einstellbar sein und es soll einen Zugverlauf geben,
der es erlaubt, bereits durchgeführte Züge wieder rückgängig zu machen.
Weiterhin soll das gesamte Design des Spiels den Spezifikationen des
Android-Material-Designs entsprechen und dabei intuitiv bleiben.

\section{Kartenspiele}
\sectionauthor{\frank}

Zusätzlich zu dem Schachspiel sollen Kartenspiele existieren, um die Applikation
abzurunden und Abwechslung zum Schach zu bieten. Hier werden drei Spiele
implementiert.
