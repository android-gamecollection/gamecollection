\chapter{Implementierung}

\section{Chess}
\sectionauthor{\oliver}

Schach ist eins der bekanntesten und gleichzeitig eins der anspruchsvollsten Spiele der Welt. Aufgrund der Komplexität
und benötigten Weitsichtigkeit schaffte es erst 1996 der Schachcomputer "Deep Blue" vom IBM den damalig amtierenden
Schachweltmeister Garro Kasparow zu besiegen. Heutzutage existieren viele Implementiereungen fähiger
Schachprogramme und KI`s. In diesem Projekt wurde wurde "Carballo" genommen.\unsure{Kommt das hier überhaut hin? Wenn ja, mehr?}

\subsection{Das Spielfeld}

Für das klassische karierte Spielfeld wurde die Klasse \code{CheckeredGameboardView} erstell,
welche wie der Name schon sag von der Androidklasse \code{View} erbt. Haputbestandteil ist ein
Zweidimensionales Array aus Androids\code{Rect}, welche die einzelnen Felder des Spielfelds darstellen.
Diese werden nach Aufruf von \code{onSizeChanged} der Größe des Displays angepasst und je nach
Einstellung um die Stärke des gewünschten Randes verschoben, sodass auf jedem Gerät ein identisches
Spielerlebnis erzegut werden kann. Um bei einem Touch auf die View zu ermitteln, auf welches der
Felder getippt wurde setzt die Methode \code{getSquareFromTouch(int x, int y)} die in \code{Rect}
mitgelieferte Funktion \code{contains(int x, int y)} ein und gibt die Array-Koordinaten des gesuchten
Kastens zurrück. Bei der Colorierung und Markierung der Felder bezieht sich die Klasse auf die in den
Einstellungen gespeicherten Werte.

\subsection{Der Chesswrapper}

In der Welt der Informatik sucht man den Begriff \emph{"Langlebig"} vergeblich, permanent werden
Module und Codeabschnitte verändert und ausgetauscht. Auch bei der Spielsammlung sind solche
Modifikationen vorgekommen und werden wohl in absehbarer Zeit wieder passieren. Aus diesem Grund
ist die Schachlogik nur über eine einzige Schnittstelle zugänglich, dem \code{ChessWrapper}. Dieser
umschließt alle benötigten Funktionen der verwendeten Schachbibliothek und erleichtert das Austauschen
der selbigen beachtlich. Neben grundlegenden Funktionen wie das ausgeben der aktuellen Figurenaufstellung
und das setzen von Schachzügen beinhaltet der Wrapper auch die Funktionen der künstlichen Intelligenz, welche
sich in der selben Bibliothek befindet. In zukünftigen Versionen werden diese Funktionen getrennt behandelt,
um allgemein geltenden Codemetriken gerecht zu werden.\unsure{Soll das so bleiben oder kommt das so rüber als ob wir das schlecht gemacht haben?}